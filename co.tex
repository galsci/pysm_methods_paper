\textbf{Giuseppe:} 
PySM3 employs models of Carbon Monoxide (CO) emission  of  the first 3 CO rotational lines, i.e.    $J = 1\rightarrow0, 2\rightarrow1$, and $3\rightarrow2$ transitions at 115.3, 230.6, and 345.8\,GHz, respectively
 We adopted  the CO $J = 1\rightarrow0$ \texttt{Type-1}   map  released   by \emph{Planck } collaboration  \cite{}.  This CO map  is obtained exploiting the mismatches in the detector bandpass to recover the CO from the CMB and the other Galactic foregrounds, with the MILCA component separation algorithm.  
 
 Although the    \texttt{Type-1} maps suffer of a lower SNR  with respect to the other maps .they are less affected by Galactic foreground residuals, as they encode information coming solely from the 100 GHz \emph{Planck} detectors. 
 Moreover, being the  CO map has been released at the nominal resolution (10  arcmin) we convolve it with a 1 deg gaussian beam to further lower the noise contamination  especially at intermediate and high Galactic latitudes. 


CO line emission can be linearly polarized via the Goldreich-Kylafis effect \citep{Goldreich1981, Crutcher2012}.  As CO polarization surveys are hard to be realized,  mainly due to the intrinsically low degree of polarization and the long integration time required to achieve a significant detection,  so far the approach has been to model it assuming an, albeit small, degree of polarization. \citet{Puglisi2016a} presented a model to simulate the polarized emission of CO lines in molecular clouds at high Galactic latitudes by considering the 3D spatial distribution of CO in the Galaxy. The model   successfully reproduced not only the angular power spectrum of the observed \emph{Planck}  CO intensity maps \citepalias{Planck_2013_XIII},  but also the intensity profile  in longitude bins at low Galactic latitudes.  

Given the lack of observational constraints, the model assumed a strong correlation between the CO polarization and the polarized galactic dust emission to forecast the amplitude of CO polarized emission.